\documentclass[11pt]{article}
\usepackage{amsmath,amssymb,amsthm,geometry,graphicx}
\usepackage{bm}
\usepackage{titlesec}
\usepackage{hyperref}
\geometry{margin=1in}

\title{Exact Four-Fold Circulation from 4D$\to$3D Vortex Projection}
\author{Trevor Norris}
\date{\today}

\newtheorem{theorem}{Theorem}

\begin{document}
\maketitle

\begin{abstract}
When a codimension-2 vortex object in four spatial dimensions is observed on a three-dimensional slice, what is the precise amplification of its measured circulation? We prove an exact \emph{four-fold} enhancement under a clear set of geometric and regularity assumptions. The result is purely geometric: it depends on the projection/slicing of a 4D vortex sheet (or filamentary limit) across the $w=0$ hyperplane and \emph{not} on any model-specific coupling constants. The 3D signal decomposes into four equal contributions: (i) direct intersection with the slice, (ii) projection from the $w>0$ half-space, (iii) projection from the $w<0$ half-space, and (iv) a continuity-induced (``drainage'') contribution that enforces conservation as axial flux passes through the slice. Each piece contributes exactly one base circulation $\Gamma$; hence the total is $4\Gamma$. We present the assumptions explicitly, give a compact derivation using the projected kernel identity $\int_0^{\infty}\!dw\,(\rho^2+w^2)^{-3/2}=1/\rho^2$, and provide error bounds for finite thickness and curvature: the result holds as $\xi/\rho\to0$ with corrections of order $O((\xi/\rho)^2+(\kappa\rho)^2)$. Two worked geometries (straight pierce and tilted disk) illustrate the invariance of the $4\times$ factor. The theorem is accompanied by plain-language analogies and units/dimensions checks to support independent symbolic and numerical verification.
\end{abstract}

\paragraph{Keywords} vortex, codimension-2 defect, dimensional reduction, circulation, differential forms, Biot--Savart kernel, projection

\section{Introduction}
\textbf{Problem.} Higher-dimensional vortex structures (codimension-2 in 4D) are natural in field theories and generalized fluid models. When such an object is observed on a 3D slice (e.g., an experimental or numerical ``measurement space''), how large is the induced 3D circulation compared to the intrinsic circulation carried by the 4D object? We show the answer is an \textit{exact factor of four}.

\textbf{What is new.} Prior literature discusses anomalous numerical factors in other contexts (convention choices, topological multiplicities), but an \emph{exact, kernel-level} proof that a 4D vortex sheet projects to $4\times$ the base 3D circulation under slicing appears absent. Our contribution is a clean decomposition into four equal pieces with an explicit integral identity and verifiable error bounds.

\textbf{Intuition.} Imagine a thin 4D ``membrane'' carrying circulation that slices through the 3D world like a skewer through a soap film. On the film, you see (1) the circulation where the skewer pierces, plus (2) influence from the membrane just above, (3) from just below, and (4) a compensating swirl the film must set up to let fluid drain across the puncture while keeping loop-circulation accounting consistent. Four separate $1\times$ contributions add to $4\times$ the base.

\textbf{Scope.} We keep the paper geometric: no application-specific constants; applications are deferred to an Outlook paragraph.

\section{Setup, Assumptions, and Notation}
\subsection{Coordinates and objects}
Let $(x,y,z,w)\in\mathbb R^4$ denote 4D Euclidean coordinates. The observation slice is the hyperplane $\Pi:=\{w=0\}\cong\mathbb R^3$ with in-slice position $\bm r=(x,y,z)$ and distance $\rho=\|\bm r-\bm r_0\|$ from a chosen local intersection point $\bm r_0\in\Pi$. A smooth 2D vortex sheet $\Sigma\subset\mathbb R^4$ intersects $\Pi$ transversely along a smooth curve $\mathcal C_0\subset\Pi$. In a sufficiently small neighborhood we may flatten $\Sigma$ (straightening map) and work with a canonical local model.

\subsection{Velocity fields and circulation}
Let $\bm v_4(x,y,z,w)$ be a 4D velocity field, smooth away from $\Sigma$, with in-slice restriction $\bm v(\bm r):=\bm v_4(\bm r,0)$. Circulation of the intersection is measured by any small closed loop $\gamma_\rho\subset\Pi$ linking $\mathcal C_0$ once:
\[
\oint_{\gamma_\rho} \bm v\cdot d\bm\ell = \Gamma,\qquad \xi\ll\rho\ll R.
\]
Here $\xi$ is a thickness/regularization scale (Sec.\,\ref{sec:errors}), $R$ a curvature scale of $\mathcal C_0$.

\subsection{Assumptions checklist}
\begin{itemize}
  \item \textbf{Thin-slice limit:} $\xi/\rho\to0$.
  \item \textbf{Local flattening:} In a small neighborhood, $\Sigma$ is approximated by a plane (straightening map error $O((\kappa\rho)^2)$, $\kappa$ curvature).
  \item \textbf{Reflection symmetry:} Local profile is even in $w$ about $w=0$ (or deviations are higher order).
  \item \textbf{Regular decay:} Induced fields decay sufficiently fast so the integrals below converge; see the kernel identity.
  \item \textbf{Steady near the slice:} For the continuity (drainage) argument we assume a steady thin-slice regime; unsteady corrections are higher order.
\end{itemize}

\subsection{Units and dimensions}
Velocity $[\bm v]=L/T$; circulation $[\Gamma]=L^2/T$; lengths $[\rho]=[w]=L$. All kernel manipulations preserve these dimensions.

\section{Projected kernel and the core identity}
The azimuthal component on $\Pi$ induced by a sheet element at axial offset $|w'|$ decays as
\begin{equation}
\label{eq:kernel}
K(\rho,w')=\frac{1}{(\rho^2+w'^2)^{3/2}},\qquad \rho>0.
\end{equation}
The half-space integral obeys the exact identity
\begin{equation}
\label{eq:halfintegral}
\int_0^{\infty}\!\frac{dw'}{(\rho^2+w'^2)^{3/2}}=\frac{1}{\rho^2},\qquad \int_{-\infty}^{\infty}\!\frac{dw'}{(\rho^2+w'^2)^{3/2}}=\frac{2}{\rho^2}.
\end{equation}
A convenient antiderivative is $w'/(\rho^2\sqrt{\rho^2+w'^2})$. Equation~\eqref{eq:halfintegral} underlies the exact one-$\Gamma$ contributions in the next section.

\paragraph{Units check.} $K\sim L^{-3}$; multiplying by $\rho^2\,dw'\sim L^3$ yields dimensionless weight; the prefactor that carries units is $\Gamma/(2\pi\rho)\sim L/T$, giving a velocity contribution as required.

\paragraph{Intuition.} If you collapse the 4D neighborhood onto the slice, each ring at fixed $w'$ contributes an azimuthal ``twist'' whose strength decays like distance$^{-3}$. Integrating all rings in one half-space yields exactly the same azimuthal speed as the in-plane base contribution.

\section{Main theorem and four lemmas}
\begin{theorem}[Four-fold projection]
Under the assumptions above, the circulation measured on any small linking loop $\gamma_\rho\subset\Pi$ in the thin-slice limit decomposes into four equal parts, each equal to $\Gamma$:
\[
\oint_{\gamma_\rho}\bm v\cdot d\bm\ell\;=\;\underbrace{\Gamma}_{\text{direct}}+\underbrace{\Gamma}_{\text{upper}}+\underbrace{\Gamma}_{\text{lower}}+\underbrace{\Gamma}_{\text{drainage}}\;=\;4\,\Gamma.
\]
The result is exact as $\xi/\rho\to0$ and admits the error bound in Sec.\,\ref{sec:errors}.
\end{theorem}

We prove the theorem via four lemmas.

\subsection*{Lemma 1 (Direct intersection = $\Gamma$)}
Let $\gamma_\rho\subset\Pi$ link $\mathcal C_0$ once. By definition of the sheet’s strength and 4D Stokes on a small spanning ribbon in $\Pi$,
\[
\oint_{\gamma_\rho}\bm v\cdot d\bm\ell = \iint_{S_\rho}(\nabla\times\bm v)\cdot d\bm S=\Gamma.
\]
\textit{Analogy.} This is the ``what you see is what you get'' part—the slice directly cuts through the vortex and measures its own $1\times$ circulation.

\subsection*{Lemma 2 (Upper half-space = $\Gamma$)}
In the thin-slice/local model, the azimuthal speed increment induced on $\Pi$ by elements at $w'>0$ is
\[
\delta v_\theta(\rho;w')=\frac{\Gamma}{2\pi\rho}\,\frac{\rho^2}{(\rho^2+w'^2)^{3/2}}\,dw'.
\]
Integrating $w'\in(0,\infty)$ and using \eqref{eq:halfintegral} gives $\int_0^{\infty}\delta v_\theta=\Gamma/(2\pi\rho)$. Multiplying by $2\pi\rho$ to convert to circulation,
\[
\oint_{\gamma_\rho}\bm v^{(+)}\cdot d\bm\ell = \Gamma.
\]
\textit{Analogy.} This is the contribution of everything ``just above'' the slice; the distance-decay integrates to exactly one base unit.

\subsection*{Lemma 3 (Lower half-space = $\Gamma$)}
By reflection symmetry (even kernel in $w$), the identical calculation yields
\[
\oint_{\gamma_\rho}\bm v^{(-)}\cdot d\bm\ell = \Gamma.
\]
\textit{Analogy.} Mirror of Lemma 2 from below.

\subsection*{Lemma 4 (Continuity-induced swirl = $\Gamma$)}
Consider a pillbox volume that straddles $\Pi$ with caps at $w=\pm\varepsilon$ and side boundary a cylinder of radius $\rho$. In the steady thin-slice limit, axial flux through the caps balances in-slice divergence, so the scalar potential part of $\bm v$ accommodates radial ``drainage.'' To maintain the measured loop circulation $\oint_{\gamma_\rho}\bm v\cdot d\ell$ consistent with Lemma 1 while axial flux crosses the slice, a solenoidal (curl) component is induced whose circulation equals the axial flux. By the local normalization of $\Gamma$ this flux equals $\Gamma$, hence
\[
\oint_{\gamma_\rho}\bm v^{(\mathrm{drain})}\cdot d\bm\ell=\Gamma.
\]
\textit{Analogy.} Like water draining through a pinhole in a membrane: continuity forces a ring current so that the ``accounting identity'' for loop circulation holds even as fluid crosses the film.

\paragraph{Forms framing.} Let $u$ be the velocity 1-form and $\Omega=du$ the vorticity 2-form on $\mathbb R^4$. Take a small 3-chain $M$ whose boundary splits into four pieces adapted to the geometry: (i) an in-plane ribbon spanning $\gamma_\rho$, (ii) a cap in $w>0$, (iii) a cap in $w<0$, (iv) a side surface that encodes the drainage constraint. Applying Stokes, $\int_{\partial M}\Omega=0$. Locality and symmetry show the four boundary contributions are equal and each equals the flux label $\Gamma$. Mapping these pieces back to 3D observables yields Lemmas 1–4 without double-counting.

\section{Error bounds and robustness}\label{sec:errors}
Let $\xi$ be the regularization scale in $w$ and $\kappa$ the curvature of $\mathcal C_0$. For $\xi\ll\rho\ll \kappa^{-1}$:
\begin{itemize}
  \item \textbf{Tail truncation.} If one truncates the half-space integral at $|w'|\le L$, the discarded tail contributes at most $O((\rho/L)^2)$.
  \item \textbf{Finite thickness.} Replacing an ideal $\delta$-profile by an even mollifier of width $\xi$ changes each hemisphere’s contribution by $O((\xi/\rho)^2)$.
  \item \textbf{Curvature.} Straightening-map errors scale as $O((\kappa\rho)^2)$ and affect the azimuthal speed at the same order.
\end{itemize}
Hence
\[
\oint_{\gamma_\rho}\bm v\cdot d\bm\ell=4\,\Gamma\,\Big[1+O\!\big((\xi/\rho)^2+(\kappa\rho)^2\big)\Big].
\]

\section{Worked geometries}
\subsection{Straight pierce (axisymmetric local model)}
Flatten $\Sigma$ so its normal is along $w$; the intersection curve is a straight line segment through $\Pi$. Cylindrical symmetry about the normal gives the kernel \eqref{eq:kernel}. Lemmas 1–3 follow exactly; Lemma 4 is enforced by the pillbox balance. This is the canonical calculation and matches the theorem.

\subsection{Tilted disk}
Parameterize a small disk patch of $\Sigma$ tilted by a small angle relative to $\Pi$. The straightening map reduces the patch to the axisymmetric case up to errors $O((\kappa\rho)^2)$. Because the hemisphere integration is taken along $w$ lines normal to $\Pi$, the tilt induces only higher-order corrections in the azimuthal component; the four contributions remain $\Gamma$ each within the stated error bound.

\section{Relation to prior work (brief)}
Our result is geometric and kernel-based. Existing discussions of ``factor-of-4'' effects in other areas (e.g., convention choices, topological multiplicities) do not supply a \emph{circulation} theorem under 4D$\to$3D slicing. Likewise, while codimension-2 defects are well studied and numerical studies of 4D vortices exist, the exact 4$\times$ projection law with explicit kernel identity and error control appears unproven elsewhere. This paper fills that gap.

\section{Outlook (applications without constants)}
Because the theorem is agnostic to constitutive details, any theory where an observable on a 3D slice linearly couples to a 4D circulation source can inherit the $4\times$ enhancement under the same assumptions. Potential arenas include generalized fluid models, superfluid vortex observations via tomographic slices, and effective field theories with one large extra spatial dimension. Specific couplings and measurement protocols belong in separate domain papers.

\appendix
\section*{Appendix A: Core integrals and bounds}
For $\rho>0$,
\[
\int_0^{\infty}\!\frac{dw'}{(\rho^2+w'^2)^{3/2}}=\frac{1}{\rho^2},\quad
\int_{-\infty}^{\infty}\!\frac{dw'}{(\rho^2+w'^2)^{3/2}}=\frac{2}{\rho^2}.
\]
Tail bound for truncation at $L>0$:
\[
\int_L^{\infty}\!\frac{dw'}{(\rho^2+w'^2)^{3/2}}=\frac{1}{\rho^2}-\frac{L}{\rho^2\sqrt{\rho^2+L^2}}\le \frac{1}{2L^2}.
\]
Even mollifiers of width $\xi$ alter the half-space integral by $O((\xi/\rho)^2)$ (two integrations by parts because the kernel has two bounded moments).

\section*{Appendix B: Differential-forms framing (sketch)}
Let $u$ be the velocity 1-form on $\mathbb R^4$, $\Omega=du$ the vorticity 2-form. Away from $\Sigma$, $\Omega=0$. Choose a small 3-chain $M$ whose boundary decomposes into four pieces adapted to the geometry: in-plane ribbon, upper cap, lower cap, and side/drainage surface. Then $\int_{\partial M}\Omega=0$. Locality and symmetry show each boundary piece contributes the same flux label $\Gamma$; mapping these contributions to in-slice circulation recovers the four lemmas.

\section*{Appendix C: Regularization and thickness}
Replace the idealized sheet by a profile even in $w$ with width $\xi$; all results persist with corrections $O((\xi/\rho)^2)$. Curvature enters through the straightening map; errors are $O((\kappa\rho)^2)$.

\section*{Appendix D: Numerical verification plan}
To assist independent checks: (i) choose a smooth even profile in $w$ (e.g., compactly supported bump or $\mathrm{sech}^2$), (ii) perform half-space quadrature for Lemma 2 and 3 contributions and compare with $\Gamma$, (iii) verify convergence of the sum to $4\Gamma$ as $\xi/\rho\to0$, (iv) repeat for a tilted patch to validate curvature scaling.

\end{document}

