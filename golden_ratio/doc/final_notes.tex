\section{Methods: AI-Assisted Derivation and Verification}

This work demonstrates a novel methodology for rigorous mathematical physics research using large language models (LLMs) as collaborative tools, bringing software engineering verification practices to theoretical physics. The approach emphasizes systematic cross-validation, bias prevention, and comprehensive testing—principles that ensure mathematical correctness independent of the tools used.

\subsection{Collaborative Framework}
The research employed four distinct LLMs with complementary strengths: Claude Opus for conceptual development and logical reasoning, Grok for mathematical derivations and paper structure, GPT for independent verification, and Claude Sonnet for systematic cataloging and automated testing. This multi-model approach prevents single-source bias and enables cross-validation at each step.

\subsection{Iterative Verification Protocol}
The derivation process followed a structured protocol:

\begin{enumerate}
\item \textbf{Initial Development}: Physical intuition was formalized into mathematical statements through iterative dialogue, with each major claim checked by at least two independent LLMs.

\item \textbf{Mathematical Cataloging}: After drafting each section, Sonnet systematically cataloged every equation, derivation step, dimensional analysis, and unit conversion into a comprehensive list.

\item \textbf{Automated Verification}: For each cataloged item, SymPy scripts were generated to verify:
   \begin{itemize}
   \item Algebraic correctness of all derivations
   \item Dimensional consistency throughout
   \item Numerical accuracy of specific values (e.g., $\varphi = (1+\sqrt{5})/2$)
   \item Behavior of functions (convexity, critical points)
   \end{itemize}

\item \textbf{Error Resolution}: When verification failed, the specific issue was presented to Grok and GPT for independent solution proposals. Sonnet then evaluated these proposals before implementation, preventing cascading errors.

\item \textbf{Bias Prevention}: After corrections, a fresh session with Sonnet re-cataloged the mathematics from scratch, preventing any carryover of previous assumptions. This clean-room approach ensured each verification stood independently.

\item \textbf{Completeness Check}: The final catalog was cross-verified by having Opus and Grok independently list all mathematical claims, then comparing their outputs for completeness.
\end{enumerate}

\subsection{Empirical Validation}
Beyond internal consistency, the framework required empirical validation:

\begin{enumerate}
\item Multiple LLMs independently proposed numerical tests the theory should satisfy
\item These proposals were consolidated and implemented as additional SymPy verifications
\item Key predictions (energy minimum at $\varphi$, robustness bounds) were tested numerically
\item Visualizations were generated to confirm theoretical predictions matched computational results
\end{enumerate}

\subsection{Reproducibility and Transparency}
All verification scripts, both for mathematical consistency and numerical validation, are publicly available at \url{https://github.com/trevnorris/papers}. The repository includes:
\begin{itemize}
\item Complete SymPy verification of every equation in the paper
\item Numerical tests of key theoretical predictions
\item Scripts for generating all figures
\end{itemize}

% --- References (placeholders) ---
\begin{thebibliography}{99}

\bibitem{Saffman1992}
P. G. Saffman,
\emph{Vortex Dynamics}
(Cambridge University Press, Cambridge, 1992).

\bibitem{Fetter2009}
A. L. Fetter,
``Rotating trapped Bose–Einstein condensates,''
\emph{Rev. Mod. Phys.} \textbf{81}, 647--691 (2009).

\bibitem{Widnall1972}
S. E. Widnall,
``The stability of a helical vortex filament,''
\emph{J. Fluid Mech.} \textbf{54}, 641--663 (1972).

\bibitem{Hardin1982}
J. C. Hardin,
``The velocity field induced by a helical vortex filament,''
\emph{Phys. Fluids} \textbf{25}, 1949--1952 (1982).

\bibitem{FukumotoOkulov2005}
Y. Fukumoto and V. L. Okulov,
``The velocity field induced by a helical vortex tube,''
\emph{Phys. Fluids} \textbf{17}, 107101 (2005).

\bibitem{Betchov1965}
R. Betchov,
``On the curvature and torsion of an isolated vortex filament,''
\emph{J. Fluid Mech.} \textbf{22}, 471--479 (1965).

\bibitem{KleinMajdaDamodaran1995}
R. Klein, A. J. Majda, and K. Damodaran,
``Simplified equations for the interaction of nearly parallel vortex filaments,''
\emph{J. Fluid Mech.} \textbf{288}, 201--248 (1995).

\bibitem{HirthLothe}
J. P. Hirth and J. Lothe,
\emph{Theory of Dislocations}, 2nd ed.
(Wiley, New York, 1982).

\bibitem{khesin2021}
B. Khesin and H. Wang,
``The golden ratio and hydrodynamics,''
arXiv:2104.02225 [physics.flu-dyn] (2021).

\bibitem{mokry2008}
P. Mokry,
``Critical merger distance of two co-rotating Rankine vortices,''
\emph{J. Fluid Mech.} \textbf{611}, 1--22 (2008).

\bibitem{lin2021}
C.-J. Lin and L. Zou,
``Reaction-diffusion dynamics in a Fibonacci chain: Interplay between classical and quantum behavior,''
arXiv:2103.14044 [cond-mat.stat-mech] (2021).

\bibitem{kauffman2004}
L. H. Kauffman and S. J. Lomonaco,
``Braiding operators are universal quantum gates,''
\emph{New J. Phys.} \textbf{6}, 134 (2004).

\bibitem{wang2024}
Y. Wang et al.,
``Superconductivity in the Fibonacci chain,''
arXiv:2403.06157 [cond-mat.supr-con] (2024).

\bibitem{sun2023}
M. Sun et al.,
``Enhancement of superconductivity in the Fibonacci chain,''
arXiv:2307.05009 [cond-mat.supr-con] (2023).

\bibitem{matsuura2024}
M. Matsuura et al.,
``Singular continuous and nonreciprocal phonons in quasicrystal AlPdMn,''
\emph{Phys. Rev. Lett.} \textbf{133}, 136101 (2024).

\bibitem{glotzer2015}
M. Engel, P. F. Damasceno, C. L. Phillips, and S. C. Glotzer,
``Computational self-assembly of a one-component icosahedral quasicrystal,''
\emph{Nat. Mater.} \textbf{14}, 109--116 (2015).

\bibitem{li2013}
M. Li and W. Zhao,
``Essay on Kolmogorov law of minus 5 over 3 viewed with golden ratio,''
\emph{Adv. High Energy Phys.} \textbf{2013}, 680678 (2013).

\bibitem{vladimirova2021}
N. Vladimirova et al.,
``Fibonacci turbulence,''
\emph{Phys. Rev. X} \textbf{11}, 021063 (2021).

\bibitem{schewe1983}
G. Schewe,
``On the force fluctuations acting on a circular cylinder in crossflow from subcritical up to transcritical Reynolds numbers,''
\emph{J. Fluid Mech.} \textbf{133}, 265--285 (1983).

\end{thebibliography}
